% ------------------------------------------------
\StartChapter{Problem Statement}{chapter:problem-statement}
% ------------------------------------------------

This chapter formally defines the few-shot fake news detection problem and establishes the key challenges that motivate our GemGNN approach.

\textbf{Given:}
\begin{itemize}
    \item Labeled set: $\mathcal{L} = \{(x_i, y_i)\}_{i=1}^{2K}$ where $K$ examples per class
    \item Unlabeled set: $\mathcal{U} = \{x_j\}_{j=1}^{M}$ 
    \item Test set: $\mathcal{T} = \{x_k\}_{k=1}^{N}$
    \item Constraints: $K \ll M, N$
\end{itemize}

\vspace{0.3cm}

\textbf{Objective:}
\begin{itemize}
    \item Learn classifier $f: \mathcal{X} \rightarrow \{0, 1\}$ that accurately predicts labels for $\mathcal{T}$
    \item Binary classification: real news ($y = 0$) vs fake news ($y = 1$)
\end{itemize}

\vspace{0.3cm}

\textbf{Key Challenges:}
\begin{itemize}
    \item \textcolor{red}{Extreme data scarcity}: $K \in \{3 \sim 16\}$ labeled examples per class
    \item \textcolor{blue}{Content-only constraint}: No user interaction or propagation data available
\end{itemize}

\section{Problem Definition}

\textbf{Few-Shot Fake News Detection:} Given a small set of labeled news articles $\mathcal{L} = \{(x_i, y_i)\}_{i=1}^{2K}$ where $K$ represents the number of examples per class, and unlabeled news articles $\mathcal{U} = \{x_j\}_{j=1}^{M}$, learn a classifier $f: \mathcal{X} \rightarrow \{0,1\}$ that accurately predicts labels for test instances $\mathcal{T} = \{x_k\}_{k=1}^{N}$ where $K \ll M$ and $K \ll N$.

The binary classification task distinguishes between real news ($y = 0$) and fake news ($y = 1$). In few-shot scenarios, $K \in \{3, 4, 5, ..., 16\}$ labeled examples per class are available for training, creating extreme data scarcity conditions.

\section{Key Challenges}

\textbf{Data Scarcity:} With only 3-16 labeled examples per class, traditional supervised learning approaches suffer from severe overfitting and poor generalization. Standard deep learning models require thousands of examples for reliable performance.

\textbf{Privacy Constraints:} The problem explicitly excludes access to user propagation data, social network structures, or user interaction patterns. This constraint reflects real-world deployment scenarios where such data is unavailable due to privacy regulations or platform restrictions.

\textbf{Evaluation Integrity:} Existing graph-based approaches often allow information leakage between test instances during edge construction, leading to unrealistic performance estimates that do not reflect deployment conditions.

\textbf{Content-Only Detection:} Without social signals, the system must rely solely on textual content to distinguish fake from real news, requiring sophisticated semantic understanding and relationship modeling.

\section{Research Objectives}

Our research aims to develop a few-shot fake news detection system that:

\begin{itemize}
\item Achieves reliable performance with minimal labeled data ($K \leq 16$)
\item Operates without user interaction or social network data
\item Maintains realistic evaluation protocols that prevent information leakage
\item Leverages structural relationships between news articles for improved detection
\item Integrates synthetic data generation to address data scarcity challenges
\end{itemize}

These objectives drive the design of our GemGNN framework, which addresses data scarcity through synthetic interaction generation, models structural relationships via heterogeneous graphs, and ensures evaluation integrity through test-isolated edge construction strategies detailed in the methodology chapter.

% ------------------------------------------------
\EndChapter
% ------------------------------------------------