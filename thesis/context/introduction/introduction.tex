% ------------------------------------------------
\StartChapter{Introduction}{chapter:introduction}
% ------------------------------------------------

\section{Research Background and Motivation}

The proliferation of misinformation poses critical challenges to information integrity, with false news spreading significantly faster than true news on social media platforms~\cite{vosoughi2018spread}. Traditional fake news detection methods face two fundamental limitations in practical deployment: dependency on extensive labeled datasets and reliance on user propagation data that is increasingly unavailable due to privacy constraints.

Current fake news detection approaches primarily follow two paradigms: content-based analysis and propagation-based modeling. Content-based methods analyze linguistic and semantic patterns within news articles, while propagation-based approaches model information spread through social networks using user interactions and sharing patterns. However, both paradigms encounter significant limitations in real-world scenarios.

The few-shot learning challenge represents the most critical limitation, where detection systems must accurately classify news articles with minimal labeled training data. This scenario is ubiquitous when addressing emerging topics, breaking news events, or novel misinformation campaigns where extensive labeled datasets are unavailable. Traditional deep learning approaches requiring thousands of labeled examples per class fail to perform adequately in such data-scarce environments.

Propagation-based methods, despite achieving competitive performance, require comprehensive user interaction data including social network structures, user profiles, and temporal propagation patterns. Such data is increasingly difficult to obtain due to privacy regulations, platform restrictions, and the time-sensitive nature of misinformation detection. These methods also face vulnerabilities to adversarial manipulation where malicious actors can engineer propagation patterns to evade detection.

\section{Research Contributions}

This thesis presents GemGNN (Generative Multi-view Interaction Graph Neural Networks), a novel framework for few-shot fake news detection that addresses the fundamental limitations of existing approaches through synthetic data generation and heterogeneous graph neural networks. Our work makes four key contributions:

\textbf{Synthetic User Interaction Generation:} We introduce a systematic approach to generate realistic user interactions using Large Language Models (LLMs), creating heterogeneous graph structures that capture social context without requiring real user propagation data. Our method generates diverse user responses with three semantic tones (neutral, affirmative, skeptical), creating 20 synthetic interactions per news article that provide social signals while preserving privacy. This innovation enables graph-based modeling benefits without dependency on user behavior data.

\textbf{Test-Isolated Edge Construction:} We develop a principled approach to graph edge construction that prevents information leakage between training and test sets. Unlike traditional K-nearest neighbor methods that can create unrealistic connections, our test-isolated KNN strategy ensures robust evaluation protocols by constraining test nodes to connect only within their own partition. This addresses a critical limitation in graph-based few-shot evaluation where information leakage leads to overoptimistic performance estimates.

\textbf{Multi-View Graph Architecture:} We propose a multi-view learning framework that partitions DeBERTa embeddings into complementary semantic subspaces, creating multiple graph views that capture diverse content aspects. Each view constructs independent similarity-based edges, enabling the model to learn from multiple semantic perspectives simultaneously. This approach provides implicit regularization and richer representation learning in few-shot scenarios where training data is limited.

\textbf{Heterogeneous Graph Attention Networks:} We leverage existing Heterogeneous Graph Attention Networks (HAN) to effectively model relationships between news articles and synthetic user interactions. The HAN architecture employs hierarchical attention mechanisms to learn both node-level importance within relationship types and semantic-level importance across different relationship types. The framework enables transductive learning by leveraging all nodes during message passing while restricting loss computation to labeled nodes only.

Through extensive empirical evaluation, we demonstrate that standard cross-entropy loss achieves optimal performance for few-shot fake news detection, eliminating the need for complex loss engineering. Our approach achieves superior performance compared to baseline methods including traditional machine learning, transformer-based models, large language models, and existing graph-based approaches across multiple few-shot configurations on FakeNewsNet datasets.

\section{Thesis Organization}

The remainder of this thesis is organized as follows:

\textbf{Chapter 2: Problem Statement} formally defines the few-shot fake news detection problem and establishes the mathematical notation used throughout our methodology. We present the fundamental challenges and provide a rigorous problem formulation with key constraints and evaluation metrics.

\textbf{Chapter 3: Related Work} provides a comprehensive review of existing fake news detection methods, including few-shot learning strategies,traditional machine learning, language models, large language models, content-based graph-based approaches, propagation-based graph-based methods. We analyze the limitations of current approaches and position our work within the broader research landscape.

\textbf{Chapter 4: Methodology} presents the complete GemGNN framework, detailing the synthetic user interaction generation, edge construction strategies, multi-view graph architecture, and heterogeneous graph neural network design. We provide algorithmic descriptions and theoretical justifications for each component.

\textbf{Chapter 5: Experimental Setup} describes our experimental methodology, including dataset preprocessing, baseline implementations, evaluation protocols, and hyperparameter configurations. We ensure reproducibility and fair comparison across all experimental conditions.

\textbf{Chapter 6: Results and Analysis} presents comprehensive experimental results, including performance comparisons, ablation studies, and analysis of model behavior. We provide insights into the effectiveness of our approach and identify key factors contributing to performance improvements.

\textbf{Chapter 7: Conclusion and Future Work} summarizes our contributions, discusses the implications of our findings, acknowledges limitations, and outlines promising directions for future research in few-shot fake news detection.

% ------------------------------------------------
\EndChapter
% ------------------------------------------------
