% ------------------------------------------------
\StartAbstract
% ------------------------------------------------

Few-shot fake news detection remains a critical challenge in misinformation control, particularly when social propagation data is unavailable due to privacy constraints or real-time detection requirements. Traditional approaches rely heavily on extensive labeled datasets or user interaction patterns, limiting their applicability in emerging misinformation scenarios where labeled examples are scarce.

This thesis presents GemGNN (Generative Multi-view Interaction Graph Neural Networks), a novel heterogeneous graph neural network framework that addresses few-shot fake news detection without requiring real user propagation data. Our approach introduces three key innovations: (1) synthetic user interaction generation using Large Language Models to create diverse user responses with multiple semantic tones (neutral, affirmative, skeptical), enabling heterogeneous graph construction without privacy concerns; (2) test-isolated K-nearest neighbor edge construction that prevents information leakage during evaluation while maintaining graph connectivity; and (3) multi-view graph architecture that partitions DeBERTa embeddings into complementary semantic perspectives for richer representation learning.

The GemGNN framework constructs heterogeneous graphs containing news nodes and synthetic interaction nodes, connected through learned attention mechanisms in a Heterogeneous Graph Attention Network (HAN) architecture. The system operates under transductive learning where all nodes participate in message passing, but only labeled nodes contribute to loss computation. Empirical analysis demonstrates that standard cross-entropy loss achieves optimal performance, eliminating the need for complex loss engineering.

Comprehensive experiments on FakeNewsNet datasets (PolitiFact and GossipCop) across K-shot configurations (K=3\textasciitilde16) demonstrate that GemGNN consistently outperforms baseline methods including traditional machine learning approaches, transformer-based models, large language models, and existing graph-based methods. The framework achieves superior F1-scores while maintaining computational efficiency and requiring no real social interaction data.
 
The contributions establish a practical paradigm for privacy-preserving fake news detection that maintains competitive performance in few-shot scenarios through synthetic data generation and principled graph construction, making it suitable for real-world deployment where user behavior data is unavailable or restricted.

% ------------------------------------------------
\EndAbstract
% ------------------------------------------------
