% ------------------------------------------------
\StartAbstract
% ------------------------------------------------

Fake news has become a critical threat to information integrity and social stability, particularly in few-shot scenarios where limited labeled data is available for emerging topics or misinformation campaigns. Traditional fake news detection methods rely heavily on user propagation patterns or require extensive labeled datasets, making them impractical for real-world deployment where such data is scarce or unavailable due to privacy constraints. This thesis presents GemGNN (Generative Multi-view Interaction Graph Neural Networks), a novel framework for few-shot fake news detection that addresses these fundamental limitations through innovative heterogeneous graph neural network modeling.

% TODO: Add Figure 1 - GemGNN Framework Overview showing the complete pipeline from text to heterogeneous graph to HAN model

Our approach introduces four key innovations that collectively establish a new paradigm for content-based misinformation detection: First, we develop a generative user interaction simulation method using Large Language Models (LLMs) to synthesize diverse user interactions with multiple semantic tones (neutral, affirmative, skeptical), effectively creating a heterogeneous node structure that captures both content and social context without requiring real user propagation data. Second, we propose a Test-Isolated K-Nearest Neighbor (KNN) edge construction strategy that prevents information leakage between test nodes during graph construction, ensuring more realistic and robust evaluation protocols in few-shot scenarios where data leakage can lead to overoptimistic performance estimates. Third, we implement a multi-view graph construction approach that partitions news embeddings into multiple semantic perspectives, enabling the capture of diverse content aspects through complementary graph structures. Fourth, we design a specialized Heterogeneous Graph Attention Network (HAN) architecture that models complex type-specific relationships between news articles and generated user interactions through hierarchical attention mechanisms and meta-path based message passing.

% TODO: Add Table 1 - Comparison of GemGNN components with existing approaches

The GemGNN framework operates under a transductive learning paradigm where all nodes (labeled, unlabeled, and test) participate in graph message passing, but only labeled nodes contribute to loss computation, maximizing the utility of limited supervision. Our heterogeneous architecture distinguishes between news nodes (characterized by rich textual embeddings) and interaction nodes (characterized by sentiment and tone features), enabling the model to learn distinct representation spaces for different entity types while capturing their relationships through learned attention weights.

% TODO: Add Figure 2 - Heterogeneous graph structure visualization showing news nodes, interaction nodes, and edge types

Extensive experiments across comprehensive parameter grids on the FakeNewsNet datasets (PolitiFact and GossipCop) demonstrate that GemGNN significantly outperforms state-of-the-art methods across various few-shot configurations (K=3-16 samples per class). Our method achieves superior performance compared to traditional approaches (MLP, LSTM), transformer-based models (BERT, RoBERTa, DeBERTa), large language models (LLaMA, Gemma), and existing graph-based methods (Less4FD, HeteroSGT, BertGCN). Comprehensive ablation studies validate the effectiveness of each architectural component, demonstrating that the synergistic combination of generative interactions, test-isolated KNN edge construction, multi-view graph construction, and heterogeneous attention mechanisms provides substantial and consistent improvements in few-shot fake news detection performance.

% TODO: Add Table 2 - Performance comparison across different few-shot settings (K=3,8,16) on both datasets

The contributions of this work establish a new paradigm for privacy-preserving fake news detection that eliminates dependency on user behavior data while maintaining superior performance in data-scarce scenarios. The framework is particularly suitable for emerging misinformation detection tasks, privacy-sensitive applications, and scenarios where social interaction data is unavailable or unreliable, addressing critical gaps in current detection capabilities for real-world deployment.

% ------------------------------------------------
\EndAbstract
% ------------------------------------------------
