% ------------------------------------------------
\StartChapter{Conclusion and Future Work}{chapter:conclusion}
% ------------------------------------------------

This thesis presents GemGNN (Generative Multi-view Interaction Graph Neural Networks), a novel framework for few-shot fake news detection that addresses fundamental limitations of existing approaches through content-based graph neural network modeling enhanced with generative auxiliary data and rigorous evaluation protocols.

\section{Summary of Contributions}

Our work establishes several key methodological and technical contributions that advance the state-of-the-art in few-shot fake news detection:

\textbf{Heterogeneous Graph Framework:} We introduce the first systematic application of heterogeneous graph neural networks to few-shot fake news detection, creating a unified framework that models both content similarity and synthetic social interactions without requiring real user data.

\textbf{Generative User Interaction Simulation:} We develop a novel approach to synthesize realistic user interactions using Large Language Models, creating controllable synthetic social signals across multiple semantic tones (neutral, affirmative, skeptical) while maintaining complete privacy protection.

\textbf{Test-Isolated Evaluation Methodology:} We establish rigorous evaluation protocols that prevent information leakage while maintaining transductive learning benefits, ensuring realistic performance assessment for few-shot scenarios.

\textbf{Multi-View DeBERTa Architecture:} We leverage DeBERTa's disentangled attention mechanism to create embeddings with superior partitioning properties, enabling multi-view learning where each view captures distinct linguistic and semantic aspects.

\textbf{Comprehensive Framework Validation:} We demonstrate the effectiveness of our approach through extensive experiments across multiple few-shot configurations, showing consistent improvements over baseline methods including traditional machine learning, transformer-based models, and existing graph-based approaches.

\section{Key Findings}

Our experimental evaluation reveals several important insights about few-shot fake news detection:

\textbf{Heterogeneous Graph Superiority:} Heterogeneous graph structures provide substantial benefits over independent document processing by enabling specialized attention mechanisms that capture complementary information from different node and edge types.

\textbf{Synthetic Interaction Effectiveness:} LLM-generated user interactions provide meaningful signal for fake news detection, with skeptical interactions showing particularly high discriminative power and the combination of all three tones achieving optimal performance.

\textbf{Multi-View Learning Benefits:} DeBERTa embedding partitioning captures diverse semantic perspectives, with each view focusing on distinct linguistic aspects that improve model robustness when combined through learned attention mechanisms.

\textbf{Evaluation Methodology Impact:} Our test-isolated approach provides conservative but realistic performance estimates that better reflect actual deployment scenarios compared to traditional KNN methods.

\section{Limitations}

Despite significant advances, several limitations remain:

\textbf{Embedding Dependency:} Performance is fundamentally limited by the quality of underlying DeBERTa embeddings, which may miss subtle domain-specific indicators.

\textbf{Static Graph Structure:} Current approach constructs static graphs that may not capture dynamic relationships as new information becomes available.

\textbf{Sophisticated Misinformation:} Highly sophisticated fake news that closely mimics legitimate journalism style can still challenge the approach.

\section{Future Research Directions}

Based on our findings, we identify three key directions for future research:

\subsection{Inductive Learning}

\textbf{Test Node Information Isolation:} Develop methods to completely disable fetching test node information during the training process, moving from transductive to fully inductive learning. This would enable the model to generalize to completely unseen nodes without any structural information, making it more applicable to real-world scenarios where new articles arrive continuously.

\subsection{Topic and Entity Integration for Richer Feature Augmentation}

\textbf{Diverse Topic Modeling:} Create sophisticated topic extraction and entity recognition systems to establish diverse topics and entities for enhanced node connections. This would involve integrating knowledge graphs, topic models, and named entity recognition to create richer semantic relationships beyond simple content similarity.

\textbf{Knowledge-Enhanced Graphs:} Incorporate external knowledge sources to create more informed node representations and edge construction policies based on factual relationships and semantic hierarchies.

\subsection{Domain-Shifted Detection Application}

\textbf{Plug-and-Play Framework:} Develop GemGNN as a general-purpose plug-and-play solution for various anomaly detection problems beyond fake news. This includes adapting the framework for fraud detection, spam identification, and other content authenticity verification tasks.

\textbf{Cross-Domain Transfer:} Investigate how models trained on news detection can transfer to other domains with minimal adaptation, establishing GemGNN as a versatile foundation for content-based anomaly detection across diverse applications.

In conclusion, this thesis presents a significant advancement in few-shot fake news detection through the novel GemGNN framework. By establishing new paradigms for content-based detection through heterogeneous graph learning and synthetic interaction simulation, our work provides a foundation for more effective misinformation detection systems and opens clear pathways for broader applications in content authenticity verification.

% ------------------------------------------------
\EndChapter
% ------------------------------------------------
