%
% This file is part of the project of
% National Cheng Kung University (NCKU) Thesis/Dissertation Template in LaTex.
% This project is hold at
%     <https://github.com/wengan-li/ncku-thesis-template-latex>
% by Wen-Gan Li.
%
% This project is distributed in the hope of usefuling to someone,
% you can redistribute it and/or modify it under the terms of the
% Attribution-NonCommercial-ShareAlike 4.0 International.
%
% You should have received a copy of the
% Attribution-NonCommercial-ShareAlike 4.0 International
% along with this project.
% If not, see <http://creativecommons.org/licenses/by-nc-sa/4.0/legalcode.txt>.
%
% Please feel free to fork it, modify it, and try it.
% Have fun !!!
%

% ----------------------------------------------------------------------------

%\def\singlespacing{%
%    \def\default@spacing{\baselineskip=15.5pt plus .5pt minus .2pt}}
\begin{comment}
\makeatletter

% Regular text spacing
\def\doublespacing{%
    \def\default@spacing{\baselineskip=20pt plus .5pt minus .2pt}}
\def\onehalfspacing{%
    \def\default@spacing{\baselineskip=20.5pt plus .5pt minus .2pt}}
\def\singlespacing{%
    \def\default@spacing{\baselineskip=15.5pt plus .5pt minus .2pt}}
\def\specialspacing{%
    \def\default@spacing{\baselineskip=21.5pt plus .5pt minus .2pt}}

\makeatother
\end{comment}
% ----------------------------------------------------------------------------

% 設定段落之間的距離
%\setlength{\parskip}{0.3cm}
  %
  % Reset page
%  \setlength{\parindent}{1.5em}
%  \baselineskip=26pt
  %
  % Set page
  %\baselineskip=15pt
%  \setlength{\parindent}{0.0pt}
  % 設定段落之間的距離
%  \setlength{\parskip}{0.5cm}
  %

% ----------------------------------------------------------------------------

% 	延伸baseline的高度
\newcommand{\ValueDefaultLineStretch}{1.2} % Default
\newcommand{\ValueCustomLineStretch}{1.2} % Default

\newcommand{\ValueDefaultLineStretchTypeDefault}{0}
\newcommand{\ValueDefaultLineStretchTypeCustom}{1}
\newcommand{\VarDefaultLineStretchType}{%
  \ValueDefaultLineStretchTypeDefault} % Default
\newcommand{\GetDefaultLineStretchType}{%
  \VarDefaultLineStretchType}
\newcommand{\SetDefaultLineStretchType}[1]{\renewcommand{\VarDefaultLineStretchType}{#1}}

% 公開的APIs
\newcommand{\SetLineStretch}[1]%
{%
  \renewcommand{\ValueCustomLineStretch}{#1}%
  \SetDefaultLineStretchType{\ValueDefaultLineStretchTypeCustom}%
} % End of \newcommand{}

\newcommand{\UseDefaultLineStretch}
{%
  \ifthenelse{\equal{\GetDefaultLineStretchType}{\ValueDefaultLineStretchTypeDefault}}%
  {\setstretch{\ValueDefaultLineStretch}}{}%
  %
  \ifthenelse{\equal{\GetDefaultLineStretchType}{\ValueDefaultLineStretchTypeCustom}}%
  {\setstretch{\ValueCustomLineStretch}}{}%
} % End of \newcommand{}

\UseDefaultLineStretch % Default

% ----------------------------------------------------------------------------

% 過去的API, 以 Error提醒不能再使用

\newcommand{\ThesisWroteInChi}{\errmessage{模版: 由v1.4.4開始, ThesisWroteInChi已不能再使用. 請參考最新版的conf.tex使用方式.}\stop}

% ----------------------------------------------------------------------------
