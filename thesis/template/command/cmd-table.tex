%
% This file is part of the project of
% National Cheng Kung University (NCKU) Thesis/Dissertation Template in LaTex.
% This project is hold at
%     <https://github.com/wengan-li/ncku-thesis-template-latex>
% by Wen-Gan Li.
%
% This project is distributed in the hope of usefuling to someone,
% you can redistribute it and/or modify it under the terms of the
% Attribution-NonCommercial-ShareAlike 4.0 International.
%
% You should have received a copy of the
% Attribution-NonCommercial-ShareAlike 4.0 International
% along with this project.
% If not, see <http://creativecommons.org/licenses/by-nc-sa/4.0/legalcode.txt>.
%
% Please feel free to fork it, modify it, and try it.
% Have fun !!!
%

% ----------------------------------------------------------------------------

\DeclareDocumentCommand{\SetTableCaptionAndLabel}{+m +m}
{%
  \ifthenelse{\equal{#1}{\empty}}{}%
  {%
    \ifthenelse{\equal{%
      \GetStartExtendedAbstractFigureTableControl}{%
      \ValueDisableExtendedAbstractFigureTableControl}}%
      {\caption{#1}}{\caption[]{#1}}
    \ifthenelse{\equal{#2}{\empty}}{}{\label{#2}}%
  }%
} % End of \DeclareDocumentCommand{}

\DeclareDocumentCommand{\SetTableCaptionStarAndLabel}{+m +m}
{%
  \ifthenelse{\equal{#1}{\empty}}{}%
  {%
    \caption*{#1}%
    \ifthenelse{\equal{#2}{\empty}}{}{\label{#2}}%
  }%
} % End of \DeclareDocumentCommand{}

\newcommand{\DisplayTableContent}[5]
{%
  \begin{minipage}[c]{\textwidth}%
  \begin{mdframed}[skipabove=0pt, skipbelow=0pt, leftmargin=0pt, rightmargin=0pt,
    innerleftmargin=0pt, innerrightmargin=0pt, innertopmargin=0pt,
    innerbottommargin=0pt, linewidth=0pt, apptotikzsetting={%
    \tikzset{mdfbackground/.append style={opacity=#4}}}]%
  \ifthenelse{\equal{#1}{0.0}}%
  {%
    \makebox[\textwidth]%
    {%
      \setlength{\tabcolsep}{#2}%
      \renewcommand{\arraystretch}{#3}%
      #5%
    }%
  }{%
    \makebox[\textwidth]%
    {%
      \resizebox{#1\paperwidth}{!}%
      {%
        \setlength{\tabcolsep}{#2}%
        \renewcommand{\arraystretch}{#3}%
        #5%
      }%
    }%
  }%
  \end{mdframed}%
  \end{minipage}%
} % End of \newcommand{}

\pgfkeys
{
  /InsertTable/.is family, /InsertTable,
  default/.style =
  {
    scale = 0.0,
    nomtitle = \empty, % For nomenclature
    caption = \empty,
    label = \empty,
    pos = top, % Position of caption or nomtitle
    tabcolsep = 6pt,
    arraystretch = 1,
    opacity = 0.4,
  },
  scale/.estore in = \TmpValueScale,
  nomtitle/.estore in = \TmpValueNomTitle,
  caption/.estore in = \TmpValueCaption,
  label/.estore in = \TmpValueLabel,
  pos/.estore in = \TmpValuePosition,
  tabcolsep/.estore in = \TmpValueTabColSep,
  arraystretch/.estore in = \TmpValueArrayStretch,
  opacity/.estore in = \TmpValueOpacity,
} % End of \pgfkeys{}

\newcommand{\InsertTable}[2][\empty]
{%
  % Parse the input
  \pgfkeys{/InsertTable, default, #1}%
  %
  \begin{table}[H]%
  %
    \ifthenelse{\equal{\TmpValuePosition}{top}}%
    {%
      \ifthenelse{\equal{\TmpValueNomTitle}{\empty}}%
        {\SetTableCaptionAndLabel{\TmpValueCaption}{\TmpValueLabel}}%
        {\SetTableCaptionStarAndLabel{\TmpValueNomTitle}{\TmpValueLabel}}%
      \DisplayTableContent{%
        \TmpValueScale}{\TmpValueTabColSep}{%
        \TmpValueArrayStretch}{\TmpValueOpacity}{#2}%
    }%
    {%
      \DisplayTableContent{%
        \TmpValueScale}{\TmpValueTabColSep}{%
        \TmpValueArrayStretch}{\TmpValueOpacity}{#2}%
      \ifthenelse{\equal{\TmpValueNomTitle}{\empty}}%
        {\SetTableCaptionAndLabel{\TmpValueCaption}{\TmpValueLabel}}%
        {\SetTableCaptionStarAndLabel{\TmpValueNomTitle}{\TmpValueLabel}}%
    }%
  \end{table}%
} % End of \newcommand{}

% -----------------------------------------------------------------

\newcolumntype{L}[1]{>{\raggedright\let\newline\\\arraybackslash\hspace{0pt}}m{#1}}
\newcolumntype{C}[1]{>{\centering\let\newline\\\arraybackslash\hspace{0pt}}m{#1}}
\newcolumntype{R}[1]{>{\raggedleft\let\newline\\\arraybackslash\hspace{0pt}}m{#1}}

%\newcolumntype{LT}[1]{>{\raggedright\let\newline\\\arraybackslash\hspace{0pt}}p{#1}}
%\newcolumntype{LC}[1]{>{\raggedright\let\newline\\\arraybackslash\hspace{0pt}}m{#1}}
%\newcolumntype{LB}[1]{>{\raggedright\let\newline\\\arraybackslash\hspace{0pt}}b{#1}}

%\newcolumntype{CT}[1]{>{\centering\let\newline\\\arraybackslash\hspace{0pt}}p{#1}}
%\newcolumntype{CC}[1]{>{\centering\let\newline\\\arraybackslash\hspace{0pt}}m{#1}}
%\newcolumntype{CB}[1]{>{\centering\let\newline\\\arraybackslash\hspace{0pt}}b{#1}}

%\newcolumntype{RT}[1]{>{\raggedleft\let\newline\\\arraybackslash\hspace{0pt}}p{#1}}
%\newcolumntype{RC}[1]{>{\raggedleft\let\newline\\\arraybackslash\hspace{0pt}}m{#1}}
%\newcolumntype{RB}[1]{>{\raggedleft\let\newline\\\arraybackslash\hspace{0pt}}b{#1}}

% -----------------------------------------------------------------

\def\ValueTableNameDefault{Table}
\def\ValueTableNameCustom{Table} % Default
\def\UseTableNameDefault{%
  \renewcommand{\tablename}{\ValueTableNameDefault}}
\def\UseTableNameCustom{%
  \renewcommand{\tablename}{\ValueTableNameCustom}}
\newcommand{\SetCustomTableName}[1]{%
  \renewcommand{\ValueTableNameCustom}{#1}}

\UseTableNameCustom % Default

% -----------------------------------------------------------------
\begin{comment}
\def\ValueTableNameBoldOn{1}
\def\ValueTableNameBoldOff{0}
\def\VarTableNameBoldOption{\ValueTableNameBoldOn} %Default
\def\GetTableNameBoldOption{\VarTableNameBoldOption}
\newcommand{\EnableTableNameBold}{%
  \renewcommand{\VarTableNameBoldOption}{\ValueTableNameBoldOn}}
\newcommand{\DisableTableNameBold}{%
  \renewcommand{\VarTableNameBoldOption}{\ValueTableNameBoldOff}}

% ------------------------------------------

\def\ValueTableTextBoldOn{3}
\def\ValueTableTextBoldOff{2}
\def\VarTableTextBoldOption{\ValueTableTextBoldOff} %Default
\def\GetTableTextBoldOption{\VarTableTextBoldOption}
\newcommand{\EnableTableTextBold}{%
  \renewcommand{\VarTableTextBoldOption}{\ValueTableTextBoldOn}}
\newcommand{\DisableTableTextBold}{%
  \renewcommand{\VarTableTextBoldOption}{\ValueTableTextBoldOff}}
\end{comment}
% ------------------------------------------

%\newcommand\TableCaptionFormatStyleLabel[1]{#1}
%\newcommand\TableCaptionFormatStyleText[1]{\begin{bfseries}#1\end{bfseries}}
%\begin{bfseries} Text to bold \end{bfseries}
%\DeclareCaptionFormat{TableCaptionFormatStyle}{\TableCaptionFormatStyleLabel{#1#2}\TableCaptionFormatStyleText{#3}}

% Default style
\newcommand{\UseTableCaptionDefaultStyle}
{%
%  \ifthenelse{\equal{\GetTableNameBoldOption}{\ValueTableNameBoldOn}}
%  {%
%    \captionsetup[table]{labelfont=bf}
%  }%
%  {%
%    \captionsetup[table]{labelfont=normalfont}
%  }%
  %
%  \ifthenelse{\equal{\GetTableTextBoldOption}{\ValueTableTextBoldOn}}
%  {%
%    \captionsetup[table]{textfont=bf}%
%  }%
%  {%
%    \captionsetup[table]{textfont=normalfont}%
%  }%
  \captionsetup[table]{labelfont=bf, textfont=normalfont}%
} % End of \newcommand{}

% Style for Extended Abstract
\newcommand{\UseTableCaptionExtendedAbstractStyle}
{%
  \captionsetup[table]{font=bf}%
  \renewcommand{\thetable}{\arabic{table}}%
} % End of \newcommand{}

\UseTableCaptionDefaultStyle % Default

% -----------------------------------------------------------------
