% This file is need to encoded in utf-8
%
% Choose or fill in some needed information for this thesis or dissertation
% 選擇或填入你的論文一些需要使用的資料

% ----------------------------------------------------------------------------

% --- 使用的論文內容 ---
% 如果沒有打開\DemoMode
% 就會使用'./context/context.tex'中你所編寫論文內容.
% 否則會使用'./example/context.tex'的模版說明文件內容.

% \DemoMode

% ----------------------------------------------------------------------------

% --- DOI 碼 ---
% 由2018年下學期時, 論文需要插入DOI碼顯示在右下方.
% DOI碼會在國立成功大學論文上傳系統中, 在上傳時提供.
\ShowDOI{記得要改}

% ----------------------------------------------------------------------------

% --- 行距 ---
% 同學可自行設定每行的距離, 這邊是以放大縮小方式來使用.
% 所以是輸入 0.1, 0.5, 1, 1.0, 1.5, 2.0, 2 等數字.
% 預設的行距: 1.2

%\SetLineStretch{1.2}

% ----------------------------------------------------------------------------

% --- 封面上語言和名字顯示方式 ---
%
% \DisplayCoverInChi:  封面以全中文顯示
% \DisplayCoverInEng:  封面以全英文顯示
% 只能選擇其中一個, 但只有最後設定的一方有效
% 預設使用\DisplayCoverInEng
%
% 另外預設在封面上只會顯示中文或英文名字而已.
% 不論你是使用\DisplayCoverInChi或\DisplayCoverInEng,
% 使用\DisplayCoverPeoplesBothNames以設定同時顯示中英文名字.

\DisplayCoverInChi
\DisplayCoverInEng
\DisplayCoverPeoplesBothNames

% ----------------------------------------------------------------------------

% --- Title 論文題目 ---
% 填寫中文和(或)英文
% 如果題目內有必須以數學模式表示的符號,請用 \mbox{} 包住數學模式
% 如果覺得自動產生出來的題目斷行位置不適合, 可以手動加'\\'來強制斷行
% (圖書館說不管你是編寫中英混合或全英文版, 都必須同時存在中英題目)
%
% 有3種可使用, 可獨立使用, 但只有最後設定的一方有效
% \SetTitle{你的題目}{Your Title}   % 同時設定中英文題目
% \SetChiTitle{你的題目}            % 只設定中文題目
% \SetEngTitle{Your Title}         % 只設定英文題目
%
% e.g:
%
% \SetTitle %
% {國立成功大學碩博士用畢業論文LaTex模版} %
% {National Cheng Kung University (NCKU) Thesis/Dissertation Template in LaTex}
%
% or
%
% \SetChiTitle{國立成功大學碩博士用畢業論文LaTex模版}
% \SetEngTitle{National Cheng Kung University (NCKU) \\Thesis/Dissertation Template in LaTex}

\SetTitle %
{生成式多視角互動圖神經網路之少樣本假新聞檢測} %
{GemGNN: Generative Multi-view Interaction Graph Neural Networks for Few-shot Fake News Detection}

% ----------------------------------------------------------------------------

% --- Draft 初稿 ---
% 顯示 '(初稿)' (中文版) 和 '(Draft)' (英文版) 在封面
\DisplayDraft

% ----------------------------------------------------------------------------

% --- Degree name 學位 ---
%
% 有2種可選擇, 但只有最後設定的一方有效
% \PhdDegree    % 博士學位
% \MasterDegree % 碩士學位

\MasterDegree

% ----------------------------------------------------------------------------

% --- Your name 你的名字 ---
% 填寫你的中文和(或)英文

% 有3種可使用, 可獨立使用, 但只有最後設定的一方有效
% \SetMyName{你的名字}{Your name}   % 同時設定你的中英文名字
% \SetMyChiName{你的名字}           % 只設定你的中文名字
% \SetMyEngName{Your name}         % 只設定你的英文名字

\SetMyName{余振揚}{Chen-Yang Yu}

% ----------------------------------------------------------------------------

% --- Date 日期 ---

% 封面日期是統一使用學位考試合格(口試合格單)單為主要參考日期 (年、月(學位考試通過日期)).
% 例如105年7月口試,則封面日期為 中華民國105年7月 或 2016年7月.

% --- 口試的日期 ---
% 設定西元的年月日, 會自動計算出民國的年份, 和英文的月份轉換
% 次序: {年份}{月份}{日}
% \SetOralDate{2023}{12}{31}

\SetOralDate{2025}{07}{09}

%--------------------------------------------------

% --- 論文封面上的日期 ---

% 如是你是國立成功大學的學生, 則封面日期直接使用口試日期, 故不需再另設定.
% 但如果你不是國立成功大學的學生, 那本模版則不清楚 貴學校所定的規範是否要分開, 故先保留這功能.

% 設定西元的年月, 會自動計算出民國的年份, 和英文的月份轉換
% 次序: {年份}{月份}
% \SetCoverDate{2023}{12}

\SetCoverDate{2025}{07}

% ----------------------------------------------------------------------------

% --- 系所 Department or Institute ---
%
% 設定你的系所名字, e.g:
% \SetDeptMath 數學系
% \SetDeptCSIE 資訊工程學系

\SetDeptAIR

% ----------------------------------------------------------------------------

% --- 指導老師 Advisor(s) ---
% 在封面上預算了最多3位的空間
% 中文名字固定以 博士  為結尾
% 英文名字固定以 Dr. 為開頭

% 有3種可使用, 用來設定3位老師的名字
% \SetAdvisorNameX{老師的名字}{Professor's name} % 同時設定中英文名字
% \SetAdvisorChiNameX{老師的名字}                % 只設定中文名字
% \SetAdvisorEngNameX{Professor's name}         % 只設定英文名字
% (NameX為NameA, NameB, NameC)

% 使用\SetAdvisorNameA是必須的, 而如果你的指導教授有2或3位,
% 那只要增加\SetAdvisorNameB和\SetAdvisorNameC則可

\SetAdvisorNameA{李政德}{Cheng-Te Li}
% \SetAdvisorNameB{B}{B}
% \SetAdvisorNameC{C}{C}

% ----------------------------------------------------------------------------

% --- 學位考試論文證明書 Defense Certificate ---
% 使用範例版本 或 使用檔案 只能選擇其中一方

% 使用範例版本
\DisplayOralTemplate

% --- 範例版本的語言 ---
% 選擇你需要的範例
% (Only work with \DisplayOralTemplate)
% \DisplayOralChiTemplate    % 顯示中文範例版本
% \DisplayOralEngTemplate    % 顯示英文範例版本

\DisplayOralChiTemplate    % 顯示中文範例版本
\DisplayOralEngTemplate    % 顯示英文範例版本

% --- 口試委員 Committee member(s) ---
% 口試委員數量 (至少2位, 最多9位, 預設為9位)
% (Only work with \DisplayOralTemplate)
% 博士學位考試委員會置委員五人至九人
% 碩士學位考試委員會置委員三人至五人
% 口試委員人數含指導教授
\SetCommitteeSize{5}

%--------------------------------------------------

% 使用學位考試論文證明書圖片檔案
% 把你的圖片放在'context/oral'下
% 之後設定中英文版所對應是哪一個檔案
% 就算已啟用\DisplayOralImage,
% 但沒有填寫圖檔檔名的話, 都不會顯示出來.
% (例子用的'example-oral-chi.pdf'和'example-oral-eng.pdf'已放在'context/oral'中)

%\DisplayOralImage                % 顯示圖檔
%\SetOralImageChi{example-oral-chi.pdf}   % 中文版檔案
%\SetOralImageEng{example-oral-eng.pdf}   % 英文版檔案

% ----------------------------------------------------------------------------

% --- 關鍵字 Keyword ---
% 最多9個關鍵字
% 為了方便同學自行設定
% 故所產出來的PDF檔案中的關鍵字和內文摘要的關鍵字
% 可獨立個別設定

% \SetKeywords是設定所產出來的PDF中的Keyword項目
% 可同時填寫中英文
% e.g
% \SetKeywords{Keyword A (關鍵字 A)}{Keyword B (關鍵字 B)}{Keyword C (關鍵字 C)}
% 或單純中文或英文
% \SetKeywords{Keyword A}{Keyword B}{Keyword C}
% \SetKeywords{關鍵字 A}{關鍵字 B}{關鍵字 C}

\SetKeywords{Fake News Detection}{Few Shot Learning}{Transductive Learning}{Generative Interaction}{Graph Neural Network}

% 摘要中的關鍵字
% 為了方便同學們能達到以下情況:
% a. 只寫中文版摘要
% b. 只寫英文版摘要
% c. 同時寫中英文版摘要
% 故中英文版的關鍵字都是可個別設定
% \SetAbstractChiKeywords: 用來設定中文版摘要的關鍵字
% \SetAbstractEngKeywords: 用來設定英文版摘要的關鍵字
% \SetAbstractExtKeywords: 用來設定英文延伸摘要的關鍵字 (只有你要編寫英文延伸摘要才需要設定)
% 所以只要使用你需要寫的版本則可.
% 但如果2個版本都要寫, 則2個都同時使用則可.
% 沒有填寫的話, 則摘要中的關鍵字部份是不會顯示出來.
%
% e.g
% \SetAbstractChiKeywords{關鍵字 A}{關鍵字 B}{關鍵字 C}
% \SetAbstractEngKeywords{Keyword A}{Keyword B}{Keyword C}
% \SetAbstractExtKeywords{Keyword A}{Keyword B}{Keyword C}
% 英文延伸摘要的關鍵字理應會跟英文版摘要的關鍵字是一樣,
% 但為了同學能編寫不同內容和關鍵字, 故可獨立設定.

% \SetAbstractChiKeywords{國立成功大學畢業論文模版}{碩博士}{LaTex/XeLaTex}
% \SetAbstractEngKeywords{NCKU Thesis/Dissertation Template}{Graduate}{LaTex/XeLaTex}
% \SetAbstractExtKeywords{NCKU Thesis/Dissertation Template}{Graduate}{LaTex/XeLaTex}
\SetAbstractEngKeywords{Fake News Detection}{Few Shot Learning}{Transductive Learning}{Generative Interaction}{Graph Neural Network}
% ----------------------------------------------------------------------------

% --- 目錄 Index ---
% 設定可獨立使用, 但只有最後設定的一方有效

% 標題文字語言 Language
% 目錄的標題文字使用預設的中文或是英文
% \IndexChiMode:  標題文字為中文
% \IndexEngMode:  標題文字為英文
% 預設的目錄標題為: 目錄 (中文) / Table of Contents (英文)
% 預設的表格目錄標題為: 表格 (中文) / List of Tables (英文)
% 預設的圖片目錄標題為: 圖片 (中文) / List of Figures (英文)
% 預設使用\IndexEngMode

%\IndexChiMode
\IndexEngMode

% ----------------------

% 目錄標題文字 Text of title
% 如果預設文字不是你所希望的, 那可以使用這邊去個別設定你所希望的文字, 不分中英文.

% 設定目錄標題
%\SetIndexTitleText{Table of Contents / 目錄}

% 設定表格目錄標題
%\SetTablesIndexTitleText{List of Tables / 表格}

% 設定圖片目錄標題
%\SetFiguresIndexTitleText{List of Figures / 圖片}

% ----------------------------------------------------------------------------

% --- 圖片相關的設定 ---
% 預設上每一張圖的名字都是以 'Figure 2.1'
% 假如想使用自定的名字, 如 '圖 2.1'
% 則使用 \SetCustomFigureName{圖} 即可.

%\SetCustomFigureName{Figure}

% ----------------------------------------------------------------------------

% --- 表格相關的設定 ---
% 預設上每一張表的名字都是以 'Table 2.1'
% 假如想使用自定的名字, 如 '表 2.1'
% 則使用 \SetCustomTableName{表} 即可.

%\SetCustomTableName{Table}

% ----------------------------------------------------------------------------

% --- 參考文獻 Reference ---
%
% --- 使用方式 ---
%    \SetupReference{ < 設定 > }
%
% < 設定 >
%    Title (Reference的標題文字)
%    BibStyle (Reference引用時的格式)
%
% ----------------------
%
% < Title >
% 模版提供了一些預設的文字
%    \TextDefaultTitleReferenceChi: 參考文獻
%    \TextDefaultTitleReferenceEng: References
%    \TextDefaultTitleBibliographyEng: Bibliography
%
% 預設上是使用\TextDefaultTitleReferenceEng.
%
% 使用時:
%    Title={\TextDefaultTitleBibliographyEng}
%
% 或自定你的文字:
%    Title={我的參考文獻標題}
%
% ----------------------
%
% < BibStyle >
% 除非有特殊的格式要求, 否則這部份是不用管的.
%
% 使用的格式 | 作者名稱顯示的格式 | 引用時顯示的例子
%   abbrv | H. J. Simpson | [4]
%   plain | Homer Jay Simpson | [4]
%   alpha | Homer Jay Simpson | Sim95
%   apacite | Homer J. S. | Homer, 1995
%
% 模版提供了:
%   LaTex基本格式:
%       abbrv, acm, alpha, apalike, ieeetr, plain, siam, unsrt
%   可參考<https://www.sharelatex.com/learn/Bibtex_bibliography_styles>
%
%   額外的格式:
%      apacite
%   可參考<https://www.overleaf.com/latex/examples/package-example-apacite/jkssmfcpzwmy>
%
% 預設使用plain.
%
% 注意:
%   如果你要轉換使用格式時, 推薦在重新產生論文前, 先把所有除了thesis.tex外的所有
%   thesis開頭或以thesis為檔名的檔案全刪掉.
%   例如'thesis.bbl', 'thesis.aux', 'thesis.lof'等所有檔案.
%   否則有可能在產生論文時遇到錯誤, 如果遇到錯誤, 請不斷重新刪掉和重新產生論文,
%   直到解決問題為止.
%
% 已知:
%   由abbrv/plain轉去apacite必定需要刪除檔案才能進行.
% ----------------------
%
% --- 請在這邊設定你要的樣子 ---
%

%\SetupReference{%
%  Title = {\TextDefaultTitleReferenceEng},
%  BibStyle = {plain},
%} % End of \SetupReference{}

% ----------------------------------------------------------------------------

% --- 章節標題的設定 ---
% 除非對章節標題格式有任何要求, 否則這部份內容是不用管的.
%
% 模版的章節有一個預設的格式:
%
% 一般章節:
%    Chapter: Chapter 1
%    Section: 1.1
%    SubSection: 1.1.1
%    SubSubSection: (空白, 只有題目)
%
% 附錄章節:
%    Chapter: Appendix A
%    Section: A.1
%    SubSection: A.1.1
%    SubSubSection: (空白, 只有題目)
%
% 如對格式有什麼的要求, 請使用\SetNumberingFormat.

% --- 使用方式 ---
%    \SetNumberingFormat[ < 章節類型 > ]{ < 設定 > }

% ----------------------
%
% < 章節類型 >
% 針對每一種的章節都可自設自己需要的格式,
% 有8種類型提供, 包括一般章節和附錄章節.
%    Chapter (章)
%    Section (節)
%    SubSection (小節)
%    SubSubSection (小小節)
%    AppendixChapter (附錄中的章)
%    AppendixSection (附錄中的節)
%    AppendixSubSection (附錄中的小節)
%    AppendixSubSubSection (附錄中的小小節)
%
% ----------------------
%
% < 設定 >
% 以下的設定針對標題中不同內容的設定.
%    BeginText (章節號碼前面的文字)
%    EndText (章節號碼後面的文字)
%    TextAlign (標題文字的位置)
%    CNumStyle ('章' 的數字類型)
%    SNumStyle ('節' 的數字類型)
%    SSNumStyle ('小節' 的數字類型)
%    SSSNumStyle ('小小節' 的數字類型)
%    SepAtIndex (目錄中章節號碼跟章節題目中的分隔符號)
%    SepBetweenCnS ('章' 號碼跟 '節' 號碼中間的分隔符號)
%    SepBetweenSnSS ('節' 號碼跟 '小節' 號碼中間的分隔符號)
%    SepBetweenSSCnSSS ('小節' 號碼跟 '小小節' 號碼中間的分隔符號)
%
% 標題在不同位置使用的內容都不一樣:
%
% 內文:
%   <BeginText> @NUMBER@ <EndText>
%   例如: 第2章
%
%   @NUMBER@為第幾章節的那個數字
%       <CNumStyle> <SepBetweenCnS>
%         <SNumStyle> <SepBetweenSnSS>
%           <SSNumStyle> <SepBetweenSSCnSSS> <SSSNumStyle>
%   例如: 2.1, 3.1.2, A.2
%
% 目錄:
%   <BeginText> @NUMBER@ <EndText> <SepAtIndex> @TITLE@
%   例如: 第2章. 介紹
%
% 被引用時:
%   @NUMBER@
%   例如: 2.1, 3.1.2, A.2
%
% ----------------------
%
% 一個完整的 \SetNumberingFormat 的樣子:
% \SetNumberingFormat[ < 章節類型 > ]{%
%   BeginText = { @文字/符號@ }, EndText = { @文字/符號@ },%
%   TextAlign = { @Left/Center/Right@ },%
%   CNumStyle = { < 數字類型 > }, SNumStyle = { < 數字類型 > },%
%   SSNumStyle = { < 數字類型 > }, SSSNumStyle = { < 數字類型 > },%
%   SepAtIndex = { @文字/符號@ }, SepBetweenCnS = { @文字/符號@ },%
%   SepBetweenSnSS = { @文字/符號@ }, SepBetweenSSCnSSS = { @文字/符號@ },%
% } % End of \SetNumberingFormat{}
%
% ----------------------
%
% --- 數字類型 ---
%
% 模版提供以下的數字類型使用
%    ChiNum (使用 '中文數字' 方式, 如: 一二三)
%    Tiangan 使用 '天干' 方式, 如: 甲乙丙丁戊癸)
%    Arabic (使用 '阿拉伯數字' 方式, 如: 1 2 3 4 5 6)
%    LowerRoman (使用 '小寫羅馬數字' 方式, 如: i ii iii vi x)
%    UpperRoman (使用 '大寫羅馬數字' 方式, 如: I II III VI X)
%    LowerAlph (使用 '小寫英文字母' 方式, 如: a b c)
%    UpperAlph (使用 '大寫英文字母' 方式, 如: A B C)
%
% 選擇你想要的數字類型後, 在<設定>中的這些位置填寫你要的類型
%    CNumStyle
%    SNumStyle
%    SSNumStyle
%    SSSNumStyle
%
%   例如: CNumStyle={Arabic}
%
% ----------------------
%
% --- 標題文字位置 ---
%
% 模版提供以下的位置使用
%   Left: 左邊
%   Center: 置中
%   Right: 右邊
% 預設上所有章節都是Left.
%
% 例如: TextAlign={Center}
%
% ----------------------
%
% --- 例子 ---
%
% 如果 '章' 要由文字改使用為:
%        'Chapter 1' -> '第1章'
% 則使用
%   \SetNumberingFormat[Chapter]{%
%     BeginText = {第}, EndText = {章}%
%   }%
%
% -----------
%
% 如果 '附錄的章' 要由文字改使用為:
%        'Appendix A' -> '附錄 A'
% 則使用
%   \SetNumberingFormat[AppendixChapter]{%
%     BeginText = {附錄 }%
%   }%
%
% -----------
%
% 如果 '章' 要由數字改使用為:
%        'Chapter 1' -> 'Chapter -A-'
% 則使用
%   \SetNumberingFormat[Chapter]{%
%     BeginText = {Chapter -}, EndText = {-},%
%     CNumStyle = {UpperAlph},%
%    }%
%
% -----------
%
% 如果 '節' 要由數字改使用為:
%        '1.2' -> '一 -乙-'
% 則使用
%   \SetNumberingFormat[Section]{%
%     EndText = {-},%
%     CNumStyle = {ChiNum}, SNumStyle = {Tiangan},%
%     SepBetweenCnS = { -},%
%    }%
%
% -----------
%
% 如果 '節' 不想看到 '章' 的數字:
%        '1.2' -> '(2)'
% 則使用
%   \SetNumberingFormat[Section]{%
%     BeginText = {(}, EndText = {)},%
%     CNumStyle = {}, SNumStyle = {Arabic},%
%     SepBetweenCnS = {},%
%    }%
% 不提供 '章' 的數字類型跟中間的分隔符號
%
% ----------------------
%
% 目錄中章節號碼跟章節題目中的分隔符號
% 正常在目錄中會顯示 'Chapter 1. ABCDEF' 或 '第一章. ABCDEF'
% 但因個人喜好, 做法不一樣, 如 'Chapter 1: ABCDEF' 或 '第一章 ABCDEF'
% 故使用 SepAtIndex 可設定你想要的符號或不需要符號
%
% 如想換'章'的由'Chapter 1. ABCDEF'換成'Chapter 1: ABCDEF'
% 則使用
%   \SetNumberingFormat[Chapter]{%
%     SepAtIndex = {:},%
%    }%
%
% 如想換'章'的由'第一章. ABCDEF'換成'第一章 ABCDEF'
%   \SetNumberingFormat[Chapter]{%
%     BeginText = {第}, EndText = {章},%
%     CNumStyle = {ChiNum},%
%     SepAtIndex = {},%
%   }%
% ----------------------
%
% --- 請在這邊設定你要的樣子 ---
%

% Chapter (章)
%\SetNumberingFormat[Chapter]{%
%  BeginText = {Chapter }, EndText = {},
%  CNumStyle = {Arabic},
%  SepAtIndex = {.},
%} % End of \SetNumberingFormat{}

% Section (節)
% \SetNumberingFormat[Section]{%
%   BeginText = {}, EndText = {},
%   TextAlign = {Left},
%   CNumStyle = {}, SNumStyle = {},
%   SepAtIndex = {}, SepBetweenCnS = {},
% } % End of \SetNumberingFormat{}

% SubSection (小節)
% \SetNumberingFormat[SubSection]{%
%   BeginText = {}, EndText = {},
%   TextAlign = {Left},
%   CNumStyle = {}, SNumStyle = {}, SSNumStyle = {},
%   SepAtIndex = {.}, SepBetweenCnS = {}, SepBetweenSnSS = {},
% } % End of \SetNumberingFormat{}

% SubSubSection (小小節)
%\SetNumberingFormat[SubSubSection]{%
%  BeginText = {}, EndText = {},
%  TextAlign = {Left},
%  CNumStyle = {}, SNumStyle = {}, SSNumStyle = {}, SSSNumStyle = {},
%  SepAtIndex = {}, SepBetweenCnS = {},
%  SepBetweenSnSS = {}, SepBetweenSSCnSSS = {},
%} % End of \SetNumberingFormat{}

% AppendixChapter (附錄中的章)
%\SetNumberingFormat[AppendixChapter]{%
%  BeginText = {Appendix }, EndText = {},
%  CNumStyle = {UpperAlph},
%  SepAtIndex = {.},
%} % End of \SetNumberingFormat{}

% AppendixSection (附錄中的節)
%\SetNumberingFormat[AppendixSection]{%
%  BeginText = {}, EndText = {},
%  TextAlign = {Left},
%  CNumStyle = {UpperAlph}, SNumStyle = {Arabic},
%  SepAtIndex = {.}, SepBetweenCnS = {.},
%} % End of \SetNumberingFormat{}

% AppendixSubSection (附錄中的小節)
%\SetNumberingFormat[AppendixSubSection]{%
%  BeginText = {}, EndText = {},
%  TextAlign = {Left},
%  CNumStyle = {UpperAlph}, SNumStyle = {Arabic}, SSNumStyle = {Arabic},
%  SepAtIndex = {.}, SepBetweenCnS = {.}, SepBetweenSnSS = {.},
%} % End of \SetNumberingFormat{}

% AppendixSubSubSection (附錄中的小小節)
%\SetNumberingFormat[AppendixSubSubSection]{%
%  BeginText = {}, EndText = {},
%  TextAlign = {Left},
%  CNumStyle = {}, SNumStyle = {}, SSNumStyle = {}, SSSNumStyle = {},
%  SepAtIndex = {}, SepBetweenCnS = {},
%  SepBetweenSnSS = {}, SepBetweenSSCnSSS = {},
%} % End of \SetNumberingFormat{}

% ----------------------------------------------------------------------------

% --- Theorems的設定 ---
% 除非對Theorems格式有任何要求, 否則這部份內容是不用管的.
%
% 提供以下的Theorems的使用:
%
%    Definition       (定義)
%    Condition        (條件)
%    Theorem          (定理)
%    Lemma            (引理)
%    Example          (例子)
%    Corollary        (推論)
%    Proposition      (主張)
%    Proof            (證明)
%    Conjecture       (猜想)
%    Note             (附註)
%    Annotation       (註解)
%    Claim            (主張)
%    Case             (情況)
%    Acknowledgment   (確認)
%    Conclusion       (結論)
%    Criterion        (標準)
%    Assertion        (斷言)
%    Problem          (問題)
%    Question         (問題)
%    Hypothesis       (假設)
%    Summary          (總結)
%
% 如對格式有什麼的要求, 請使用\SetNumberingFormat.
% 而插入新內容的話則使用\InsertXXXX.
%
% --- 使用方式 ---
%   \SetTheoremFormat[ < Theorem類型 > ]{ < 設定 > }
%     和
%   \InsertXXXX[ < 設定 > ]{ < 內容 >} (XXXX為Theorem類型), 如
%     \InsertTheorem{ abc }
%     \InsertLemma{ abc }
%     \InsertProof{ abc }
%
% ----------------------
%
% 一個完整的 \SetTheoremFormat 的樣子:
%
% \SetTheoremFormat[ < Theorem類型 > ]{%
%   ShowText = { @文字/符號@ },
%   FollowCounter = { < Counter類型 > },%
% } % End of \SetTheoremFormat{}
%
% ----------------------
%
% --- ShowText ---
%
% ShowText是指所顯示在文章中的文字. 如Proof可修改成:
%    ShowText = {證明}
%
% ----------------------
%
% --- Counter類型 ---
%
% 模版提供以下的Counter類型使用:
%
%    Section
%    Definition
%    Condition
%    Theorem
%    Lemma
%    Example
%    Corollary
%    Proposition
%    Proof
%    Conjecture
%    Note
%    Annotation
%    Claim
%    Case
%    Acknowledgment
%    Conclusion
%    Criterion
%    Assertion
%    Problem
%    Question
%    Hypothesis
%    Summary
%
%  有一些Theorem是不需要Counter, 故那些是不會需要這設定. 如Proof/Note.
%  而需要的則全預設跟隨Section Counter.
%
%  以下為預設使用Counter的清單:
%
%    Definition
%    Condition
%    Theorem
%    Lemma
%    Example
%    Corollary
%    Proposition
%    Conjecture
%    Criterion
%    Assertion
%    Problem
%    Question
%    Hypothesis
%
% ----------------------
%
% --- 請在這邊設定你要的樣子 ---
%

%\SetTheoremFormat[Definition]{ShowText = {Definition}}%
%\SetTheoremFormat[Condition]{ShowText = {Condition}}%
%\SetTheoremFormat[Problem]{ShowText = {Problem}}%
%\SetTheoremFormat[Example]{ShowText = {Example}}%
%\SetTheoremFormat[Theorem]{ShowText = {Theorem}}%
%\SetTheoremFormat[Lemma]{ShowText = {Lemma}}%
%\SetTheoremFormat[Corollary]{ShowText = {Corollary}}%
%\SetTheoremFormat[Proposition]{ShowText = {Proposition}}%
%\SetTheoremFormat[Conjecture]{ShowText = {Conjecture}}%
%\SetTheoremFormat[Proof]{ShowText = {Proof}}%
%\SetTheoremFormat[Note]{ShowText = {Note}}%
%\SetTheoremFormat[Annotation]{ShowText = {Annotation}}%
%\SetTheoremFormat[Claim]{ShowText = {Claim}}%
%\SetTheoremFormat[Case]{ShowText = {Case}}%
%\SetTheoremFormat[Acknowledgment]{ShowText = {Acknowledgment}}%
%\SetTheoremFormat[Conclusion]{ShowText = {Conclusion}}%
%\SetTheoremFormat[Criterion]{ShowText = {Criterion}}%
%\SetTheoremFormat[Assertion]{ShowText = {Assertion}}%
%\SetTheoremFormat[Question]{ShowText = {Question}}%
%\SetTheoremFormat[Hypothesis]{ShowText = {Hypothesis}}%
%\SetTheoremFormat[Summary]{ShowText = {Summary}}%

% ----------------------------------------------------------------------------
