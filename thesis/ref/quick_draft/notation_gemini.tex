% --- notation.tex ---
% A curated notation table for the GemGNN presentation.
% This table is designed to be concise, precise, and logically structured.
\documentclass{article}
\usepackage{xcolor}
\usepackage{amsmath}
\usepackage{booktabs}
\usepackage{array}
\usepackage{multirow}
\usepackage{graphicx}
\usepackage{caption}
\usepackage{subcaption}
\usepackage{amsmath}
\usepackage{amssymb}

\begin{document}

\begin{table}[h!]
\centering
\renewcommand{\arraystretch}{1.5} % Increases vertical spacing for readability
\begin{tabular}{l p{0.7\textwidth}}
\toprule
\textbf{Notation} & \textbf{Description} \\
\midrule

% --- Group 1: Problem Formulation ---
% These symbols define the core machine learning task.
\multicolumn{2}{l}{\textit{Problem Formulation}} \\
\addlinespace[0.3em] % Adds a small vertical space after the heading
$\mathcal{L}, \mathcal{U}, \mathcal{T}$ & Labeled, unlabeled, and test sets of news articles. \\
$K$ & Number of labeled examples per class (K-shot). \\
$y_i, \hat{y}_i$ & Ground-truth and predicted label for news node $n_i$. \\
\addlinespace[0.6em]

% --- Group 2: Graph Representation ---
% These symbols describe the fundamental data structure used.
\multicolumn{2}{l}{\textit{Graph Representation}} \\
\addlinespace[0.3em]
$G=(V, E)$ & A heterogeneous graph with nodes $V$ and edges $E$. \\
$V_n, V_i$ & Sets of news nodes and synthetic interaction nodes. \\
$\mathbf{x}_v \in \mathbb{R}^d$ & Initial $d$-dim feature vector for a node $v$ (DeBERTa). \\
$\mathbf{h}_v$ & Learned representation (hidden state) of node $v$. \\
\addlinespace[0.6em]

% --- Group 3: Key GemGNN Concepts ---
% These symbols highlight the novel contributions of the GemGNN framework.
\multicolumn{2}{l}{\textit{Key GemGNN Concepts}} \\
\addlinespace[0.3em]
$I_i$ & Set of synthetic user interactions for a news node $n_i$. \\
$k$ & Number of nearest neighbors in KNN graph construction. \\
$\mathcal{V}$ & Number of views in the multi-view architecture. \\
$\mathbf{h}_v^{(\nu)}$ & Node $v$'s representation in the $\nu$-th view. \\

\bottomrule
\end{tabular}
\end{table}


\end{document}